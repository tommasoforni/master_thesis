\chapter{Conclusions}

During this thesis work we developed and built two crossed Fabry-Perot optical cavities to be used in the Inverse Compton Scattering source of MariX.

The resonators are actively stabilized against the mode-locked laser source using two independent feedback systems implementing the Pound-Drever-Hall technique. The cavities have been experimentally characterized, regarding their mirror reflectivities, their Finesse, their spatial and spectral coupling with the external laser mode, and their PSD frequency noise.

A new method to avoid unwanted high-power effects in the cavity, such as mirror thermal deformation and degenerate mode coupling, has been theoretically developed and an experimental feasibility test has been conducted, giving positive results. 

We developed a technique to move the waists of the cavities while keeping them stabilized with the laser, to be used in BriXS to generate dual-color X-rays for K-edge subtraction imaging. The experimental results indicate that this solution is feasible, yielding a movement of 80\,$\mu$m in less than 100\,ms.

The main cavity and its stabilization system has been continually upgraded during the work, regarding for example the mountings, the electronics and the mirrors. The PSD frequency noise has been used as benchmark for this upgrades, since we want the lowest possible power fluctuation inside the cavity. The latest cavity model has a Finesse of about 3000, corresponding to a linewidth of 33\,kHz, with an integrated frequency noise of about 3500\,Hz.

The next steps in the R\&D program consist in developing and testing new piezo-driven mirror mountings for both the compensation method and the focus shift method, building a hood enclosure to further reduce the noise and developing a reliable method to identify and damp the mechanical oscillators in the system. Two other R\&D lines are also currently starting: one dedicated to the development of the amplification system, and one dedicated to the RF-guns for the electron bunch production. In particular after the amplification system the power in the cavity will reach high enough levels to directly observe the mentioned high power effects.