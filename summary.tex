\chapter*{Abstract}

The MariX project consists in the development of two new X-ray sources: a high-repetition rate Free Electron Laser (whom applications are in ultra-fast linear spectroscopy, cristallography, pump-probe experiments and spectroscopy) and an Inverse Compton Scattering source (with applications in medical research and radiotherapy, cultural heritage and material science). In particular the ICS source, called BriXS, will also be used for K-edge subtraction (KES) contrast imaging, a diagnostic method exploiting the sharp change in a intravenous contrast agent used for example in cancer detection.

The ICS process requires high-photon flux to produce high energy (up to 180\,keV) X-ray photons, this can be achieved using a Fabry-Perot cavity, an enchantment resonator. KES imaging requires two different X-ray energies to be produced in temporal proximity and the Compton process presents a dependence on the interaction angle that can be exploited for this dual-color X-ray production. The solution that we propose consists in using two crossed Fabry-Perot optical cavities, switching between the two by the means of moving mirrors, in order to generate photons at both 32\,keV and 34\,keV, the energies needed for iodine KES.

In this thesis work, we built two crossed 4-mirrors bow-tie Fabry-Perot resonators and stabilized them against a mode-locked laser source using the Pound-Drever-Hall (PDH) technique. The cavities have then been characterized, measuring important parameters such as the cavity Finesse, the power spectral density noise, the mirror reflectivities and the spatial and spectral coupling with the laser beam. The latest cavity model reaches a Finesse of about 3000 and an integrated frequency noise of 3500\,Hz, giving a relative frequency fluctuation between cavity and laser of the order of $10^{-11}$. The method to switch the cavity interacting with the electron beam has been studied and experimentally tested, it consists in moving the cavities foci in and out of the interaction point using piezo driven mirrors while the cavity remains stabilized with the source. We obtained a movement of 80\,$\mu$m in under 100\,ms, indicating the feasibility of the method. Also a technique for dealing with high power effects such as thermal deformation of the mirrors and degenerate mode coupling losses has been developed and tested, again with positive results.

In the first chapter an overview of the MariX project is given, with attention on the laser system that drives both the FEL and ICS sources. In the second chapter I explain the theoretical concepts necessary to understand the experimental results, they concern mainly the inverse Compton scattering process and the physics of optical resonators and laser beams. Theoretical predictions are also given, regarding the expected results and possible obstacles in future developments. The third chapter is a brief study on the cavities stabilization systems and the PDH technique, given its importance in the experimental apparatus. In the fourth chapter, I present the experimental setup and results obtained during the thesis work. Finally, in the conclusions I highlight the next steps to be taken in the R\&D program.

